%%%%%%%%%%%%%%%%%%%%%%%%%%%%%%%%%%%%%%%%%
% Twenty Seconds Resume/CV
% LaTeX Template
% Version 1.1 (8/1/17)
%
% This template has been downloaded from:
% http://www.LaTeXTemplates.com
%
% Original author:
% Carmine Spagnuolo (cspagnuolo@unisa.it) with major modifications by 
% Vel (vel@LaTeXTemplates.com)
%
% License:
% The MIT License (see included LICENSE file)
%
%%%%%%%%%%%%%%%%%%%%%%%%%%%%%%%%%%%%%%%%%

%----------------------------------------------------------------------------------------
%	PACKAGES AND OTHER DOCUMENT CONFIGURATIONS
%----------------------------------------------------------------------------------------

\documentclass[letterpaper]{twentysecondcv} % a4paper for A4

%----------------------------------------------------------------------------------------
%	 PERSONAL INFORMATION
%----------------------------------------------------------------------------------------

% If you don't need one or more of the below, just remove the content leaving the command, e.g. \cvnumberphone{}

\profilepic{../figures/foto}
\cvname{Oscar Ruiz\\ Cigarrillo} % Your name
\cvjobtitle{Maestro en Ciencias \\ Aplicadas} % Job title/career

\cvdate{13 de julio de 1993} % Date of birth
\cvaddress{San Luis Potos\'i, M\'exico} % Short address/location, use \newline if more than 1 line is required
\cvnumberphone{+52 4442384382} % Phone number
\cvsite{https://github.com/RUCO13} % Personal website
\cvmail{ruizoscar.1393@gmail.com} % Email address

%----------------------------------------------------------------------------------------

\begin{document}

%----------------------------------------------------------------------------------------
%	 ABOUT ME
%----------------------------------------------------------------------------------------

\aboutme{} % To have no About Me section, just remove all the text and leave \aboutme{}

%----------------------------------------------------------------------------------------
%	 SKILLS
%----------------------------------------------------------------------------------------

% Skill bar section, each skill must have a value between 0 an 6 (float)
\skills{{Bash/5},{Julia/5},{Lua/5},{\LaTeX/6},{C/6},{C++/6},{Fortran/6},{Python/6},{Linux/6}}

\lang{{Espa\~nol/6},{Ingl\'es/5}}

%------------------------------------------------

% Skill text section, each skill must have a value between 0 an 6
%\skillstext{{lovely/4},{narcissistic/3}}

%----------------------------------------------------------------------------------------

\makeprofile % Print the sidebar



\section{Educaci\'on}

\begin{twenty} % Environment for a list with descriptions
	\twentyitem{Desde 2017}{Candidato a Doctor en Ciencias Aplicadas (Fot\'onica)}
	                       {\\Universidad Aut\'onoma de San Luis Potos\'i, M\'exico}
	                       {\emph{``Study of optical anisotropy in coupled quantum wells, a novel source unperturbed systems''}}
	\twentyitem{2015-2017}{Maestr\'ia en Ciencias Aplicadas (Fot\'onica)}
	                      {\\Universidad Aut\'onoma de San Luis Potos\'i, M\'exico}
	                      {\emph{``Crecimiento y Caracterizaci\'on de Microcavidades \'Opticas \\ Semiconductoras''}}
	\twentyitem{2011-2015}{Ingenier\'ia F\'isica}
	                      {\\Universidad Aut\'onoma de San Luis Potos\'i, M\'exico}
	                      {Promedio: 9.2}
	
	%\twentyitem{<dates>}{<title>}{<location>}{<description>}
\end{twenty}



%----------------------------------------------------------------------------------------
%	 PUBLICATIONS
%----------------------------------------------------------------------------------------

\section{Publicaciones}

\begin{twentyshort} % Environment for a short list with no descriptions
	\twentyitemshort{2017}{{\color{blue}``Optical detection of graphene nanoribbons synthesized on stepped SiC surfaces''}\\
	                        L.F. Lastras-Mart{\'\i}nez, J. Almendarez-Rodr{\'\i}guez, G. Flores-Rangel, N.A. Ulloa-Castillo, {\color{black} O. Ruiz-Cigarrillo}, C.A. Ibarra-Becerra , R. Castro-Garc{\'\i}a, R.E. Balderas-Navarro, M.H. Oliveira Jr and  J.M.J Lopes, \textit{Journal of Applied Physics} \textbf{122(3)},  035701, (2017)}


   	\twentyitemshort{2017}{{\color{blue}``Microscopic optical anisotropy of exciton-polaritons in a GaAs-based semiconductor microcavity''}\\
   	L.F. Lastras-Mart{\'\i}nez, E. Cerda-M\'endez, N.A. Ulloa-Castillo,R. Herrera-Jasso, L. E. Rodríguez-Tapia, {\color{black} O. Ruiz-Cigarrillo}, R. Castro-Garc\'ia, K. Biermann, P. V. Santos. \textit{Physical Review B},\textbf{ 2017, vol. 96, no 23, p. 235306}}
   
   	\twentyitemshort{2019}{{\color{blue}``Differential reflectance contrast technique in near field limit: Application to graphene''}\\
   	 L.F. Lastras-Mart{\'\i}nez, D. Medina-Escobedo, G. Flores-Rangel, R.E. Balderas-Navarro, {\color{black} O. Ruiz-Cigarrillo}, R. Castro-Garc{\'\i}a,
   	 M. del P. Morales-Morelos, J. Ortega-Gallegos, M. Losurdo. \textit{ AIP Advances},\textbf{ 2019, vol. 9, no 4, p. 045309}}
    
    \twentyitemshort{2021}{{\color{blue}``Optical contrast in the near-field limit for structural characterization of graphene nanoribbons''}\\
    	 G. Flores-Rangel, L.F. Lastras-Mart{\'\i}nez, R. Castro-Garc{\'\i}a, {\color{black} O. Ruiz-Cigarrillo}, R.E. Balderas-Navarro, L.D.Espinosa-Cuellar A.Lastras-Martínez, J.M.J.Lopes. \textit{ AIP Advances},\textbf{ Volume 536, 15 January 2021, 147710}}
     
    \twentyitemshort{2021}{{\color{blue}``Optical anisotropies of asymmetric double GaAs (001) quantum wells''}\\
     {\color{black}O. Ruiz-Cigarrillo}, L. F. Lastras-Mart\'{\i}nez, E. A.
     Cerda-M\'endez, et al. \textit{Physical Review B},\textbf{ 2021, vol. 103, no 3, p. 035309.}}

	 \twentyitemshort{2022}{{\color{blue}`Photoluminescence of double quantum wells: asymmetry and excitation laser wavelength effects''}\\
	 C. A. Bravo-Velázquez,L. F. Lastras-Mart\'{\i}nez{\color{black}O. Ruiz-Cigarrillo}, et al. \textit{physica status solidi (b)},\textbf{  2022, vol. 259, issue 4.}}
   
	%\twentyitemshort{<dates>}{<title/description>}
\end{twentyshort}

\section{Distinciones}

\begin{twentyshort} % Environment for a short list with no descriptions
	\twentyitemshort{2017}{Tercer lugar nivel investigación en el concurso de carteles Enseñanza e Investigación 2017 "Fis. Candelario Pérez Rosales", XVII Semana}
	\twentyitemshort{2018}{Tercer lugar nivel investigación en el concurso de carteles Enseñanza e Investigación 2018 "Fis. Candelario Pérez Rosales", XVIII Semana}
	\twentyitemshort{2022}{Segundo lugar nivel enseñanza en el concurso de carteles Enseñanza e Investigación 2022 "Fis. Candelario Pérez Rosales", XVIII Semana}
	\twentyitemshort{2022}{Primer lugar nivel investigación en el concurso de carteles Enseñanza e Investigación 2022 "Fis. Candelario Pérez Rosales", XVIII Semana}
	%\twentyitemshort{<dates>}{<title/description>}
\end{twentyshort}
\newpage 
\makeprofile 
%----------------------------------------------------------------------------------------
%	 AWARDS
%----------------------------------------------------------------------------------------
%----------------------------------------------------------------------------------------
%	 EXPERIENCE
%----------------------------------------------------------------------------------------

\section{Experiencia}

\begin{twenty} % Environment for a list with descriptions
	\twentyitem{2017}{ Variable Compleja, Electromagnetismo}
	{\\UASLP}
	{Profesor Asistente}
	
	\twentyitem{2017}{ Creación y Edición de Documentos Científicos en LaTeX: Curso Básico}
	{UASLP}
	{Curso}
	
	\twentyitem{2017}{ F\'isica}
	{\\Nivel Secundaria}
	{Docente}
	
	\twentyitem{2019-2022}{C\'alculo (Diferencia e Integral), Física I, Física II, Probabilidad y Estadística, Matemáticas I (Álgebra), Matemáticas III (Geometría Analítica)  }
	{\\Nivel Medio Superior}
	{Docente}
	
	\twentyitem{2019-2022}{Programaci\'on, Sistemas operativos, C\'alculo (Diferencial e Integral), Física I, Física II, Probabilidad y Estadística, Álgebra Superior  }
	{\\Nivel Superior}
	{Docente}
	
\end{twenty}



\section{Participaci\'on en conferencias nacionales}

\begin{twenty} % Environment for a list with descriptions
	\twentyitem{2016}{Congreso Nacional de Física. }
	                 {\\Crecimiento y Caracterización de Microcavidades Ópticas de (Al,Ga)As.}
                     {Póster}
                     
   	\twentyitem{2017}{X Reunión Anual de la División de Información Cuántica}
   {\\ Avances en el crecimiento de microcavidades III-V para condensados cuánticos en estado sólido}
   {Póster}
  

\end{twenty}



\begin{twenty}
	  	\twentyitem{2017}{Congreso Nacional de Física}
	{\\ Crecimiento y caracterización óptica in-situ y en tiempo real de microcavidades de (Al,Ga)As}
	{Ponencia}
	
\twentyitem{2018}{Reunión Anual de la División de Estado Sólido}
	{\\ Detección de Excitones indirectos en Pozos Cuánticos Acoplados Mediante Fotorreflectancia}
	{Ponencia}
	
\twentyitem{2019}{Reunión Anual de la División de Estado Sólido}
	{\\Estudio de Excitones Indirectos y Triones en pozos cuánticos asimétricos acoplados}
	{Póster}
	%\twentyitem{<dates>}{<title>}{<location>}{<description>}
\end{twenty}



\section{Intereses}\\
\textbf{Investigaci\'on:}\\
En la investigación mis principales intereses son en el area de física del estado solido y materia condensada, tanto experimentalmente como en el desarrollo de código libre para realizar cálculos numéricos. Principalmente, las espectroscopias para el estudio de propiedades ópticas en semiconductores, así como el desarrollo de código para el análisis y modelos numéricos para la interpretación  física. \\

\textbf{Docencia:}\\
Mis intereses en la docencia es promover el uso e implementación de herramientas basas en código libre, para el desarrollo de la labor docente. Como docente mi principal objetivo es fomentar el espíritu y curiosidad sobre el aprendizaje de las ciencias. 

\end{document} 
